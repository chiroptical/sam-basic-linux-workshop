%\PassOptionsToPackage{gray}{xcolor}
\documentclass[hyperref={pdfpagelabels=false},12pt]{beamer}
\setbeamertemplate{frametitle}[default][center]
\mode<presentation>
{
 \usetheme{default}      % or try Darmstadt, Madrid, Warsaw, ...
 \usecolortheme{default} % or try albatross, beaver, crane, ...
 \usefonttheme{default}  % or try serif, structurebold, ...
 \setbeamertemplate{navigation symbols}{}
 \setbeamertemplate{caption}[numbered]
} 

\usepackage[utf8]{inputenc}
\usepackage{mathtools}
\usepackage{bm}
\usepackage{helvet}
\usepackage{listings}
\usepackage{gensymb}
\usepackage{array}
\usepackage{times}
\usepackage{xcolor}
\usepackage{default}
\usepackage{ulem}
\usepackage{minted}

%\usetheme{Pittsburgh}

\xdefinecolor{darkgreen}{rgb}{0.11,0.64,0.22}
\title[Basic Unix/Linux]{{Basic Linux}}
\author[Basic Unix/Linux]{{\textbf{Barry}, Fangping, Kim, Ketan}}
\date{}

\beamertemplatenavigationsymbolsempty

\begin{document}
\begin{frame}[plain]
\titlepage
\end{frame}

\begin{frame}
\frametitle{Schedule}
\centering
\begin{itemize}
    \item Today: Basic Linux
    \item : SaM Cluster Usage
    \item : Linux Power Tools
    \item : Python Programming
    \item : R Programming
\end{itemize}
\end{frame}

\begin{frame}{What will you learn today?}
    \begin{itemize}
        \item Quick introduction to Linux command line tools relating to:
        \begin{itemize}
            \item general purpose utilities
            \item navigating file systems
            \item handling files
            \item the linux environment
            \item filtering data
            \item remote operations and copying files
        \end{itemize}
        \item Present the tools you will use during ``Linux Power Tools''
    \end{itemize}
\end{frame}

\begin{frame}{What do we expect from you?}
    \begin{itemize}
        \item You have access to our cluster, or a command line
        \item You are willing to learn and experiment on the command line
        \item Basics about code snippets:
        \begin{itemize}
            \item Lines which begin with `\$' are commands with, or without, arguments
            \item All other lines are output from those commands
        \end{itemize}
    \end{itemize}
\end{frame}

\begin{frame}
\frametitle{General-purpose utilities}
\centering
\inputminted[bgcolor=lightgray,linenos,fontsize=\footnotesize]{bash}{code/general-purpose-utilities-1.txt}
\end{frame}

\begin{frame}
\frametitle{General purpose utilities cont.}
\centering
\inputminted[bgcolor=lightgray,linenos,fontsize=\footnotesize]{bash}{code/general-purpose-utilities-2.txt}
\end{frame}

\begin{frame}
\frametitle{Navigate the Filesystem}
\centering
%cd, pwd, mkdir, find, df, du
\inputminted[bgcolor=lightgray,linenos,fontsize=\footnotesize]{bash}{code/navigate-the-filesystem-1.txt}
\end{frame}

\begin{frame}
\frametitle{Navigate the filesystem cont.}
\centering
\inputminted[bgcolor=lightgray,linenos,fontsize=\footnotesize]{bash}{code/navigate-the-filesystem-2.txt}
\end{frame}

\begin{frame}
\frametitle{Navigate the filesystem cont.}
\centering
\inputminted[bgcolor=lightgray,linenos,fontsize=\footnotesize]{bash}{code/navigate-the-filesystem-3.txt}
\end{frame}

\begin{frame}
\frametitle{Handling files}
\centering
%less, more, cat, wc, chmod, chown, umask, tar, gzip, gunzip
\inputminted[bgcolor=lightgray,linenos,fontsize=\footnotesize]{bash}{code/handling-files-1.txt}
\end{frame}

\begin{frame}
\frametitle{The Environment}
\centering
aliases, ps, export, history, PS1, 
\end{frame}

\begin{frame}
\frametitle{Simple Filters}
\centering
head, tail, sort, column, uniq, tr, cut, paste 
\end{frame}

\begin{frame}
\frametitle{More filters}
\centering
awk, sed, grep
\end{frame}

\begin{frame}
\frametitle{The Shell}
\centering
wildcards, escaping and quoting, pipe and redirection, tee, variables
\end{frame}

\begin{frame}
\frametitle{Remote Connectivity and Networks}
\centering
ssh, scp, rsync, ftp, curl
\end{frame}

\begin{frame}
\frametitle{Useful keyboard shortcuts}
\centering
ctrl-l, ctrl-e, ctrl-a, 
\end{frame}

\begin{frame}
\frametitle{Understanding the Unix Command: Locate commands; internal vs. external}
\centering
\begin{itemize}
    \item \mintinline{sh}{apropros}
    \item \mintinline{sh}{locate}
\end{itemize}
\end{frame}

\begin{frame}
\frametitle{Exercises}
\begin{enumerate}
    \item Print system information on cluster (hint slide: )
    \item Print your username using environment variables to the command line
    (hint slide: )
    \item Why was I getting the wrong answer for $\pi*\sqrt{2}$? Hint: use the
    man page, look for ``scale'' under the ``VARIABLES'' section. The definition
    of scale is located in the ``NUMBERS'' section. Don't forget ``/'' searches.
    \item What is $\pi$ using the formula $\pi=4\times\arctan(1)$? Hint: use
    man page to activate the ``standard math library'' and determine the
    $\arctan$ function. Don't forget ``/'' searches!  \item How long does the
    calculation take?
\end{enumerate}
\end{frame}

\begin{frame}
\frametitle{Exercises cont.}

\end{frame}

\end{document}

