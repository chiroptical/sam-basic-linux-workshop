%\PassOptionsToPackage{gray}{xcolor}
\documentclass[hyperref={pdfpagelabels=false},12pt]{beamer}
\setbeamertemplate{frametitle}[default][center]
\mode<presentation>
{
 \usetheme{Warsaw}      % or try Darmstadt, Madrid, Warsaw, ...
 \usecolortheme{default} % or try albatross, beaver, crane, ...
 \usefonttheme{default}  % or try serif, structurebold, ...
 \setbeamertemplate{footline}[frame number]
 \setbeamertemplate{caption}[numbered]
} 

\usepackage[utf8]{inputenc}
\usepackage{mathtools}
\usepackage{bm}
\usepackage{helvet}
\usepackage{listings}
\usepackage{gensymb}
\usepackage{array}
\usepackage{times}
\usepackage{xcolor}
\usepackage{default}
\usepackage{ulem}
\usepackage{minted}
\usepackage{hyperref}

% Great Commands
\newcommand{\ig}[2]{\includegraphics[width=#1\linewidth]{#2}}
\newcommand{\mybutton}[2]{\hyperlink{#1}{\beamerbutton{{#2}}}}
\newcommand{\myvbutton}[2]{\vfill\hyperlink{#1}{\beamerbutton{{#2}}}}

% Color Scheme
\definecolor{pittblue}{RGB}{28,41,87}
\definecolor{pittgold}{RGB}{205,184,125}
\setbeamercolor{structure}{fg=pittgold}
\setbeamercolor{button}{bg=pittblue}

\xdefinecolor{darkgreen}{rgb}{0.11,0.64,0.22}
\title[Basic Unix/Linux]{{Basic Linux}}
\author[Basic Unix/Linux]{{\textbf{Barry}, Fangping, Kim, Ketan}}
\date{}

\beamertemplatenavigationsymbolsempty

\begin{document}

\begin{frame}[label=started]{Getting Started}
    \begin{itemize}
        \item Download the slides from: \url{https://pitt.box.com/v/basic-unix-slides}
        \item Log into the cluster, or
        \item Download git:
        \begin{itemize}
            \item Windows: install from \url{https://git-scm.com/downloads}
            \item Mac: install from \url{https://git-scm.com/downloads}, port, or brew
            \item Linux: install from package manager
        \end{itemize}
        \item Open Git bash (Windows) or Terminal Emulator (Mac/Linux)
    \end{itemize}
\inputminted[bgcolor=lightgray,linenos,fontsize=\scriptsize]{bash}{code/getting-started.txt}
\myvbutton{navigate}{a clickable button}
\end{frame}

\begin{frame}[plain]
\titlepage
\end{frame}

\begin{frame}{Schedule}
\begin{itemize}
    \item Today: Basic Linux
    \item Wednesday 10/12 (9am -- 11am): SaM Cluster Usage
    \item Thursday 10/27 (10am -- 12pm): Linux Power Tools
    \item Thursday 11/10 (1:30pm -- 3:30pm): Python Programming
    \item Thursday 12/1 (8:00am -- 12:00pm): R Programming
\end{itemize}
\end{frame}

\begin{frame}{What will you learn today?}
    \begin{itemize}
        \item Quick introduction to Linux command line tools relating to:
        \begin{itemize}
            \item general purpose utilities
            \item navigating file systems
            \item handling files
            \item the linux environment
            \item filters
            \item remote operations and copying files
        \end{itemize}
        \item Present the tools and concepts for bash scripting
    \end{itemize}
\end{frame}

\begin{frame}{What do we expect from you?}
    \begin{itemize}
        \item You have access to our cluster, or a command line
        \item You are following along with the code
        \item Code snippets:
        \begin{itemize}
            \item Lines which begin with `\$' are commands with, or without, arguments
            \item Lines which begin with `\#' are comments
            \item All other lines are output from those commands
        \end{itemize}
        \item Mantras:
        \begin{itemize}
            \item Tab completion is your friend
            \item You are always typing too much
            \item If you plan to do it more than once, write a script
        \end{itemize}
    \end{itemize}
\end{frame}

\begin{frame}[label=general-1]{General-purpose utilities}
\inputminted[bgcolor=lightgray,linenos,fontsize=\footnotesize]{bash}{code/general-purpose-utilities-1.txt}
\myvbutton{exercises-1}{exercises}
\end{frame}

\begin{frame}[label=general-2]{General purpose utilities cont.}
\inputminted[bgcolor=lightgray,linenos,fontsize=\footnotesize]{bash}{code/general-purpose-utilities-2.txt}
\myvbutton{exercises-1}{exercises}
\end{frame}

\begin{frame}[label=navigate]{Navigate the Filesystem}
\mybutton{started}{change directory}
\inputminted[bgcolor=lightgray,linenos,fontsize=\footnotesize]{bash}{code/navigate-the-filesystem-1.txt}
\end{frame}

\begin{frame}{The find command}
\inputminted[bgcolor=lightgray,linenos,fontsize=\footnotesize]{bash}{code/navigate-the-filesystem-2.txt}
\end{frame}

\begin{frame}{How much space?}
\inputminted[bgcolor=lightgray,linenos,fontsize=\footnotesize]{bash}{code/navigate-the-filesystem-3.txt}
\end{frame}

\begin{frame}{Handling files}
%less, more, cat, wc, chmod, chown, umask, tar, gzip, gunzip
\inputminted[bgcolor=lightgray,linenos,fontsize=\footnotesize]{bash}{code/handling-files-1.txt}
\end{frame}

\begin{frame}{Handling files: Permissions}
\inputminted[bgcolor=lightgray,linenos,fontsize=\footnotesize]{bash}{code/handling-files-2.txt}
\begin{center}
    \ig{0.75}{images/permissions.png}
\end{center}
\inputminted[bgcolor=lightgray,linenos,fontsize=\footnotesize]{bash}{code/handling-files-3.txt}
\end{frame}

\begin{frame}[label=environment]{The Environment}
%aliases, ps, export, history, PS1
\inputminted[bgcolor=lightgray,linenos,fontsize=\footnotesize]{bash}{code/the-environment-1.txt}
\myvbutton{exercises-1}{exercises}
\end{frame}

\begin{frame}{Aliasing -- Shortcuts}
\inputminted[bgcolor=lightgray,linenos,fontsize=\footnotesize]{bash}{code/the-environment-2.txt}
\begin{itemize}
    \item \color{red}{Don't use these for ssh!}
    \item \color{red}{Don't use these in scripts!}
\end{itemize}
\end{frame}

\begin{frame}{History}
\inputminted[bgcolor=lightgray,linenos,fontsize=\footnotesize]{bash}{code/the-environment-3.txt}
\begin{itemize}
    \item Good reference: \url{http://www.thegeekstuff.com/2011/08/bash-history-expansion/}
\end{itemize}
\end{frame}

\begin{frame}{Filters: head, tail}
%head, tail, sort, column, uniq, tr, cut, paste 
%awk, sed, grep
\begin{itemize}
    \item Filters are Linux power tools
\end{itemize}
\inputminted[bgcolor=lightgray,linenos,fontsize=\footnotesize]{bash}{code/simple-filters-1.txt}
\end{frame}

\begin{frame}{Filters: sort, uniq, tr}
\inputminted[bgcolor=lightgray,linenos,fontsize=\footnotesize]{bash}{code/simple-filters-2.txt}
\end{frame}

\begin{frame}{The Shell}
wildcards, escaping and quoting, pipe and redirection, tee, variables
\end{frame}

\begin{frame}{Remote Connectivity and Networks}
ssh, scp, rsync, ftp, curl
\end{frame}

\begin{frame}{Useful keyboard shortcuts}
ctrl-l, ctrl-e, ctrl-a, 
\end{frame}

\begin{frame}{Understanding the Unix Command: Locate commands; internal vs. external}
\begin{itemize}
    \item \mintinline{sh}{apropros}
    \item \mintinline{sh}{locate}
\end{itemize}
\end{frame}

\begin{frame}[label=exercises-1]{Exercises}
\begin{enumerate}
    \item Print system information on cluster (\mybutton{general-1}{hint})
    \item Print your username using environment variables to the command line (\mybutton{environment}{hint})
    \item Why was I getting the wrong answer for $\pi*\sqrt{2}$? Hint: use the
    man page, look for ``scale'' under the ``VARIABLES'' section. The definition
    of scale is located in the ``NUMBERS'' section. Don't forget ``/'' searches.
    (\mybutton{general-2}{hint})
    \item What is $\pi$ using the formula $\pi=4\times\arctan(1)$? Hint: use
    man page to activate the ``standard math library'' and determine the
    $\arctan$ function. Don't forget ``/'' searches! (\mybutton{general-2}{hint})
    \item How long does the calculation take? (\mybutton{general-2}{hint})
\end{enumerate}
\end{frame}

\begin{frame}{Exercises cont.}

\end{frame}

\end{document}
