%\PassOptionsToPackage{gray}{xcolor}
\documentclass[hyperref={pdfpagelabels=false},12pt]{beamer}
\setbeamertemplate{frametitle}[default][center]
\mode<presentation>
{
 \usetheme{default}      % or try Darmstadt, Madrid, Warsaw, ...
 \usecolortheme{default} % or try albatross, beaver, crane, ...
 \usefonttheme{default}  % or try serif, structurebold, ...
 \setbeamertemplate{navigation symbols}{}
 \setbeamertemplate{caption}[numbered]
} 

\usepackage[utf8]{inputenc}
\usepackage{mathtools}
\usepackage{bm}
\usepackage{helvet}
\usepackage{listings}
\usepackage{gensymb}
\usepackage{array}
\usepackage{times}
\usepackage{xcolor}
\usepackage{default}
\usepackage{ulem}
\usepackage{minted}

%\usetheme{Pittsburgh}

\xdefinecolor{darkgreen}{rgb}{0.11,0.64,0.22}
\title[Basic Unix/Linux]{{Basic Unix/Linux}}
\author[Basic Unix/Linux]{{Kim, Barry, Ketan}}
\date{}

\beamertemplatenavigationsymbolsempty

\begin{document}
\begin{frame}[plain]
\titlepage
\end{frame}

\begin{frame}
\frametitle{Getting Started}
\centering
\begin{itemize}
    \item access to cluster, VPN, etc.
    \item clone this repository
\end{itemize}
\end{frame}

%\begin{frame}
%\frametitle{How it all clicked!: A Brief History}
%\centering
%\end{frame}


\begin{frame}
\frametitle{General-purpose utilities}
\centering
\inputminted[bgcolor=lightgray,linenos,fontsize=\footnotesize]{bash}{code/general-purpose-utilities-1.txt}
\end{frame}

\begin{frame}
\frametitle{General purpose utilities cont.}
\centering
\inputminted[bgcolor=lightgray,linenos,fontsize=\footnotesize]{bash}{code/general-purpose-utilities-2.txt}
\end{frame}

\begin{frame}
\frametitle{Navigate the Filesystem}
\centering
%cd, pwd, mkdir, find, df, du
\inputminted[bgcolor=lightgray,linenos,fontsize=\footnotesize]{bash}{code/navigate-the-filesystem-1.txt}
\end{frame}

\begin{frame}
\frametitle{Navigate the filesystem cont.}
\centering
\inputminted[bgcolor=lightgray,linenos,fontsize=\footnotesize]{bash}{code/navigate-the-filesystem-2.txt}
\end{frame}

\begin{frame}
\frametitle{Navigate the filesystem cont.}
\centering
\inputminted[bgcolor=lightgray,linenos,fontsize=\footnotesize]{bash}{code/navigate-the-filesystem-3.txt}
\end{frame}

\begin{frame}
\frametitle{Handle files}
\centering
%less, more, cat, wc, chmod, chown, umask, tar, gzip, gunzip
\end{frame}

\begin{frame}
\frametitle{The Shell}
\end{frame}

\begin{frame}
\frametitle{The Environment}
\centering
aliases, ps, export, history, PS1, 
\end{frame}

\begin{frame}
\frametitle{Simple Filters}
\centering
head, tail, sort, column, uniq, tr, cut, paste 
\end{frame}

\begin{frame}
\frametitle{More filters}
\centering
awk, sed, grep
\end{frame}

\begin{frame}
\frametitle{The Shell}
\centering
wildcards, escaping and quoting, pipe and redirection, tee, variables
\end{frame}

\begin{frame}
\frametitle{Remote Connectivity and Networks}
\centering
ssh, scp, rsync, ftp, curl
\end{frame}

\begin{frame}
\frametitle{Useful keyboard shortcuts}
\centering
ctrl-l, ctrl-e, ctrl-a, 
\end{frame}

\begin{frame}
\frametitle{Useful one-liners}
\centering
cat /etc/issue
\end{frame}

\begin{frame}
\frametitle{Understanding the Unix Command: Locate commands; internal vs. external}
\centering
\begin{itemize}
    \item \mintinline{sh}{apropros}
    \item \mintinline{sh}{locate}
\end{itemize}
\end{frame}

\begin{frame}
\frametitle{Exercises}
\begin{enumerate}
    \item Print system information on cluster (hint slide: )
    \item Print your username using environment variables to the command line
    (hint slide: )
    \item What is $\pi$ to 10 digits of precision using the formula
    $\pi=4\times\arctan(1)$? Hint: use man page to activate the ``standard math
    library'' and determine the $\arctan$ function. Don't forget ``/'' searches!
    \item How long does the calculation take?
\end{enumerate}
\end{frame}

\begin{frame}
\frametitle{Exercises cont.}

\end{frame}

\end{document}

